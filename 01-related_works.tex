\textbf{Receptive field literature}
\begin{itemize}
  \item \cite{luoUnderstandingEffectiveReceptive2016}
    \item object detectors \cite{zhouObjectDetectorsEmerge2015}
      \item Interpretation of ResNet by Visualization of Preferred Stimulus in Receptive Fields \cite{kobayashiInterpretationResNetVisualization2020}
\end{itemize}
 Previous works have been done in exploring the receptive field in deep convolutional neural networks.
 \cite{luoUnderstandingEffectiveReceptive2016}  investigate the \textit{Effective Receptive Field} (ERF) of a
 neural network. ERF is smaller than the theoretical receptive field and grows as the training progresses. They also
 showed how subsampling and dilation affect the ERF. In \cite{kobayashiInterpretationResNetVisualization2020} the
 authors find that ResNets have orientation-selective neurons and double opponent colour neurons by investigating the
 preferred stimulus of the receptive field for a particular neuron. Lastly, \citep{kimDeadPixelTest2023} is the closest work
 to ours where they try to link the receptive field to the accuracy of the model. This work uses different models of the
 ResNet and WideResNet family and compares their performances and receptive fields. The main problem with this approach
 is that any change in accuracy might not be directly attributable to the receptive field given the differences in
 the number of parameters, depth and width between different models. Thus, our work measures the effect of only
 modifying the receptive. Additionally, our work shows the specific effect that modifying the receptive field has on a
 neural network loss landscape, namely, the `` ruggedness'' of the loss landscape as well as its performance in a better
 controlled environment.



\textbf{Landscape investigation}
\begin{itemize}
  \item On the Relation Between the Sharpest Directions of DNN Loss and the SGD Step Length \cite{jastrzebskiRelationSharpestDirections2019}
\end{itemize}
